%%%%%%%%%%%%%%%%%%%%%%%%%%%%%%%%%%%%%%%%%%%%%%%%%%%%%%%
%%%%%%%%%%%%%%%%%%%%%%%%%%%%%%%%%%%%%%%%%%%%%%%%%%%%%%%

% The High-Granularity Calorimeter and HGCROC

% Description of the HL-LHC phase
% What are the main changes in the CMS detector that are expected for the HL-LHC?
% Description of the High-Granularity Calorimeter
% The read-out structure and HGCROC
% Description of HGCROC
% How can you characterize the chip?
% The irradiation test: TID and SEE
% Improvements in the chip design
% Conclusion

%%%%%%%%%%%%%%%%%%%%%%%%%%%%%%%%%%%%%%%%%%%%%%%%%%%%%%%
%%%%%%%%%%%%%%%%%%%%%%%%%%%%%%%%%%%%%%%%%%%%%%%%%%%%%%%

\chapter{The CMS Endcap Calorimeter Upgrade}

After nearly fifteen years of dedicated service, the LHC will undergo a major upgrade towards the High Luminosity LHC phase (HL-LHC), which is expecting to start its operations by the end of 2029.
The upgraded machine has been designed to operate at a  centre-of-mass energy of 14~TeV and to achieve a peak instantaneous luminosity of $L=5\cdot10^{34}\;cm^{-2}s^{-1}$: in these unprecedented running conditions, a remarkable integrated luminosity of 4000~$fb^{-1}$ is expected to be collected over the anticipated ten years of data-taking. 
With the HL-LHC upgrade, the amount of collected data will significantly increase, so as the potential for new discoveries at the LHC. The increased statistics will allow for more precise measurements of the SM properties but will also improve the potential for new discoveries, enhance the sensitivity to rare processes and possibly unveil the presence of previously unknown particles and BSM scenarios.

The higher luminosities of the HL-LHC will also result in exceedingly high pile-up rates, with $\mathcal{O}(200)$ events per bunch crossing and unprecedented radiation levels, with fluences of up to $3.5\times10^6\,\textrm{s}^{-1}\,\textrm{cm}^{-2}$ and to a total absorbed dose of up to $\sim$$200\,\textrm{Mrad}$, thus posing several technical challenges for the operation of the detectors and the entire infrastructure.
In order to maintain its excellent physics performance in the high pile-up environment of the HL-LHC, the CMS Collaboration, as well as the other LHC experiments, is planning a series of major upgrades of the sub-detectors. The upgrade development and realization has already started during the Second Long Shutdown (LS2, 2018-2022) and will continue in the Third Long Shutdown (LS3, 2025-2029) when the installation and commissioning of the new detectors will be performed. 

In this chapter, after a brief overview of the CMS upgrade plans in Sec. {?}, the focus will be directed towards the High Granularity Calorimeter (HGCAL), that will replace the current endcap calorimeters and completely renovate the CMS forward calorimetry strategy. The HGCAL detector will provide the very first large-scale silicon-based imaging calorimeter employed in a high-energy physics experiment, with extremely fine transverse and longitudinal granularity, including a total of 6 million channels. Besides the unprecedented read-out segmentation, the performance requirements for the front-end electronics will be extremely stringent and require an ad-hoc electronics development. With an excellent technical effort to meet all these requirements, the HGCROC3 is the final version of the ASIC specifically designed to readout the modules of the future HGCAL. 
The characteristics of the HGCROC will be described in Sec {?} and Sec {?} will present the irradiation campaigns performed on the chip in order to prove its radiation hardness.

\section{}