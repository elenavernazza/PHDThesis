\chapter{The CMS Experiment at the LHC}

\section{The Large Hadron Collider}
\subsection{The CERN accelerator system}
\subsection{Design}
\subsection{Operations}
% Here I will talk about the HL-LHC

% By the end of Run-3, LHC is expected to have delivered a total integrated luminosity of 500
% fb−1
% to the experiments. In order to extend the discovery potential, to observe new rare events
% with small cross-sections and to perform more precise measurements on known physics processes,
% the number of collected statistics must be increased. To accomplish this goal, the LHC machine
% should offer a higher collision rate and therefore deliver higher integrated luminosity to the
% experiments. This is the main motivation for the high-luminosity upgrade of the accelerator
% which is scheduled for the years following 2025 (Fig. 1.10). The upgraded accelerator is known
% as the High-Luminosity LHC (HL-LHC).

% The HL-LHC will provide a peak luminosity of 5 × 1034 cm−2
% s
% −1 or up to 7 × 1034 cm−2
% s
% −1
% in the ultimate case. The center-of-mass energy will remain at 14 TeV as well as the bunch
% crossing frequency at 25 ns. The increase of the peak luminosity will result in an increased
% number of collisions per bunch crossing and a pile-up of 140 (or 200 in the ultimate scenario)
% events. HL-LHC is foreseen to operate for about 10 to 12 years which means that in the course
% of this time, the total integrated luminosity will reach 3000 - 4000 fb−1
% .

% The key to the increase in luminosity of the LHC is the further squeezing of the beams near
% the interaction point. New quadrupole magnets will be used in HL-LHC capable of producing a
% magnetic field up to 12 T. The magnets will be installed near the interaction points located at
% the CMS and ATLAS detectors. The properties of the superconducting material Nb3Sn will be
% exploited.
% In addition, crab cavities will be installed near the interaction points in order to reduce the
% crossing angle of the colliding beams. The beams are brought into collision at an angle of a
% few hundred microradians. This happens to prevent undesired collisions of bunches at either
% side of the interaction point since the two beams share the same vacuum chamber. However, a
% large crossing angle decreases the luminosity, as it reduces the overlap area of the bunches. In
% HL-LHC the crossing angle of the beams will be larger because the beam size will be reduced by
% a factor of two [28]. A larger crossing angle is a limiting factor for the increase in instantaneous
% luminosity. With the use of crab cavities, the head and the tail of each bunch will receive a
% kick in the opposite directions while the centroid of the bunch will receive no kick. Due to this
% deflection, the bunch overlap will be improved which increases the luminosity. A comprehensive
% overview of the physics plan and the technical upgrades for the transition to HL-LHC is given in
% [29]

\subsection{Experiments}

\section{The CMS Experiment}
\subsection{Coordinate system}
\subsection{Detector structure}

\section{The CMS trigger system}
\subsection{The Level-1 Trigger}
\subsection{The High-Level Trigger}

\section{The physics objects identification and reconstruction}
\subsection{Electrons}
\subsection{Muons}
\subsection{Taus}
\subsection{Jets}
\subsection{Missing transverse momentum}
